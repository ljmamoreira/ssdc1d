
\documentclass[11pt,twoside]{article}

\usepackage[utf8]{inputenc}
\usepackage{graphicx}
\usepackage{amsmath}
\usepackage{amssymb}
\usepackage{url}
\usepackage{graphics}
\usepackage{wasysym}
\usepackage{enumitem}
\usepackage[english,portuges]{babel}
\usepackage{color}
\usepackage[usenames,dvipsnames]{xcolor}



\begin{document}

%*********************************************************************** TITULO

\section*{Esquema de interpolação alternativo para a resolução numérica de
problemas de difusão com convecção usando o método dos volumes finitos}

\vspace{10pt}


%************************************************************************* AUTORES
Luís J.M. Amoreira$^{1}$%, Nome Sobrenome$^{1,2}$, Nome Sobrenome$^{1,2}$

\vspace{10pt}


%************************************************************************** MORADA

{
\center

    \footnotesize

$^{1}$ Departamento de Física e CMAST, Universidade da Beira Interior, Covilhã, 
Portugal \\
E-mail: amoreira@ubi.pt

%\vspace{3pt}
%
%$^{2}$ Endereço 2, Portugal \\
%Email:  email2@email.pt
%
%\vspace{3pt}
%
%$^{3}$ Endereço 1, Portugal
%
}

\vspace{25pt}

%%%%%%%%%%%%%%%%%%%%%%%%%%%%%%%%%%%%%%%%%% RESUMO


{\setlength{\parindent}{30pt}%

\small%
\section*{Resumo}
\smallskip
%%%%%%%%%%%%%%%%
%Corpo do resumo (aprox. 200 palavras)
Neste trabalho é proposto um esquema de interpolação alternativo para a
resolução de problemas de difusão com conveção e feita uma análise
comparativa das soluções que se obtêm com diferentes esquemas na resolução de um
problema unidimensional estacionário sem fontes, para o qual é possível uma
resolução analítica.

\vspace{15pt}





%%%%%%%%%%%%%%%%%%%%%%%%%%%%%%%%%%%%%%%%%%%    SECTIONS  %%%%%
\section{Introdução}
Corpo de texto regular, com referencias do estilo [1] e [2, 3]. as referências devem ser introduzidas manualmente. O estilo das referências é definido pelos editores.

\section{Esquema de ajuste contínuo}
\section{Resultados}


%%%%%%%%%%%%%%%%%%%%%%%%%%% TABELA
%\begin{table}[!h]
%\caption{Legenda da tabela.}
%
%\end{table}



%%%%%%%%%%%%%%%%%%%%%%%%%%% FIGURA
%\begin{figure}[!h]
%\centering
%\includegraphics[scale=0.95]{fig_1.pdf}
%\caption{Legenda da figura.}
%\end{figure}








%%%%%%%%%%%%%%%%%%%%%%%%%%%%%%%%%%%%%%%%%%%    SECTION  %%%%%
\section{Conclusões}

\begin{itemize}

\item Conclusão 1

\item Conclusão 2


\end{itemize}




%%%%%%%%%%%%%%%%%%%%%%%%%%%%%%%%%%%
%%
%%                 AKNOWLEDGMENTS & BIBLIOGRAPHY
%%
%%%%%%%%%%%%%%%%%%%%%%%%%%%%%%%%%%%

{\footnotesize%


%%%%%%%%%%%%%%%%%%%%%%%%%%%%%%%%%%%%%%%%%%%    AGRADECIMENTOS  %%%%%
%\section*{Agradecimentos}
% Text







%%%%%%%%%%%%%%%%%%%%%%%%%%%%%%%%%%%%%%%%%%%    REFERÊNCIAS  %%%%%
\section*{Referências}
\vspace{5pt}

\begin{enumerate}[label={[\arabic*]}]

\item	Laugksch R., Scientific Literacy: A Conceptual Overview Science Education, 84, 71-94, 2000.

\item	Driver, R., Young people's images of science, Open University Press (Buckingham and Bristol, PA) 1996.

\item	Lopes J.B., Pinto J., Silva, A. Situação formativa: um instrumento de gestão do currículo capaz de
promover literacia científica. Enseñanza de las Ciencias, Número Extra VIII Congreso Internacional sobre
Investigación en Didáctica de las Ciencias, Barcelona, pp. 1624-1629 2009.



\end{enumerate}

}


\end{document}
