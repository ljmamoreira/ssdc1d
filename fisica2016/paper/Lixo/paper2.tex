
\documentclass[11pt,twoside]{article}

\usepackage[utf8]{inputenc}
\usepackage{graphicx}
\usepackage{amsmath}
\usepackage{amssymb}
\usepackage{url}
\usepackage{graphics}
\usepackage{wasysym}
\usepackage{enumitem}
\usepackage[english,portuges]{babel}
\usepackage{color}
\usepackage[usenames,dvipsnames]{xcolor}



\begin{document}

%*********************************************************************** TITULO

\section*{Esquema de interpolação alternativo para a reso\-lução numérica de
problemas de difusão com convecção usando o método dos volumes finitos}

\vspace{10pt}


%************************************************************************* AUTORES
Luis J.M. Amoreira$^{1}$%, Nome Sobrenome$^{1,2}$, Nome Sobrenome$^{1,2}$

\vspace{10pt}


%************************************************************************** MORADA

{
\center

    \footnotesize

$^{1}$ Departamento de Física, Univeridade da Beira Interior, Covilhã,
Portugal\\
E-mail: amoreira@ubi.pt
%
%\vspace{3pt}
%
%$^{2}$ Endereço 2, Portugal \\
%Email:  email2@email.pt
%
%\vspace{3pt}
%
%$^{3}$ Endereço 1, Portugal

}

\vspace{25pt}

%%%%%%%%%%%%%%%%%%%%%%%%%%%%%%%%%%%%%%%%%% RESUMO


{\setlength{\parindent}{30pt}%

\small%
\section*{Resumo}
\smallskip
%%%%%%%%%%%%%%%%
%Corpo do resumo (aprox. 200 palavras)
Neste trabalho é proposto um esquema de discretização alternativo para a
resolução de problemas de difusão com convecção baseado no esquema de diferenças
centrais mas com coeficientes de interpolação que dependem da velocidade
convectiva.

É feita uma análise comparativa do esquema proposto com esquemas tradicionais
(diferenças centrais, \emph{upwind}, híbrido), em termos da estabilidade, da
exatidão e da taxa de convergência. O problema usado para a comparação é o da
difusão estacionária unidimensional sem fontes, num fluido incompressível em
escoamento com velocidade constante, para o qual existem soluções analíticas.

Verifica-se que o esquema proposto é estável mesmo em problemas com convecção
intensa e que produz, em geral, resultados mais aproximados da solução analítica
dos que os obtidos com os restantes esquemas considerados. Como o esquema de
difrerenças centrais, o esquema proposto é de segunda ordem em malhas homogéneas
centradas.

\vspace{15pt}





%%%%%%%%%%%%%%%%%%%%%%%%%%%%%%%%%%%%%%%%%%%    SECTIONS  %%%%%
\section{Introdução}
%Corpo de texto regular, com referencias do estilo [1] e [2, 3]. as referências
%devem ser introduzidas manualmente. O estilo das referências é definido pelos
%editores.
Na resolução numérica de equações de conservação como as que normalmente surgem
em problemas de dinâmica de fluidos usando o método dos volumes finitos,
define-se uma partição do domínio de integração com um conjunto de subdomínios
finitos chamados \emph{volumes de controle} (VC) e integram-se as equações
diferenciais a resolver em cada um desses subdomínios. Resulta deste
procedimento um sistema de equações algébricas que relacionam os valores dos
fluxos característicos do problema através das fronteiras dos vários VC. A
resolução deste sistema de equações algébricas produz os valores aproximados dos
campos a determinar nos pontos característicos de cada VC (frequentemente,
tomam-se os seus centros geométricos).  Este procedimento envolve
necessariamente a escolha de formas concretas de relacionar os fluxos nas faces
dos VC com os valores dos campos nos pontos da malha de discretização. Essas
formas concretas definem o que se chama um \emph{esquema de discretização.}

O esquema de discretização mais simples e óbvio é o esquema de diferenças
centrais (CDS) [REFS], em que os valores dos campos nas faces dos VC são
estimados por interpolação linear a partir dos valores que tomam nos dois pontos
da malha de cada VC que partilham cada face e os seus gradientes aproximados
pela fórmula de derivação central. Este esquema é de segunda ordem quando se
usam malhas uniformes e centradas, o que é notável, dada a sua simplicidade. No
entanto, em problemas envolvendo difusão e convecção, produz soluções instáveis.
Isto ocorre porque, na estimativa do valor dos campos nas faces dos VC, se dá a
mesma importância aos pontos de malha dos dois VC contíguos.  Mas parece
razoável supor que a convecção do meio material ``arraste'' parcialmente os
valores dos campos ao longo do sentido do escoamento e que, assim sendo, o
procedimento de interpolação deveria valorizar o valor do campo no VC ``a
montante'' no escoamento e desvaorizar correspondentemente o do VC ``a
jusante''.

O esquema upwind (UWS) [REFS] incorpora esta sugestão de forma radical: como
valor do campo a determinar na face entre dois volumes de controle contíguos,
assume o do VC a montante, sem tomar em linha de conta o do outro VC. Este
esquema já não apresenta os problemas de estabilidade do CDS, mas apresenta
soluções menos exatas e o seu erro é apenas de primeira ordem no passo da
discretização. Como deve ser expectável, estes problemas manifestam-se de forma
especialmente intensa nos casos em que o escoamento do meio material é lento, ou
seja, em que o transporte é dominado pela difusão. Ora, é nesse regime que o
esquema de difrerenças centrais apresenta bons resultados.





%%%%%%%%%%%%%%%%%%%%%%%%%%% TABELA
\begin{table}[!h]
\caption{Legenda da tabela.}

\end{table}



%%%%%%%%%%%%%%%%%%%%%%%%%%% FIGURA
\begin{figure}[!h]
\centering
%\includegraphics[scale=0.95]{fig_1.pdf}
\caption{Legenda da figura.}
\end{figure}








%%%%%%%%%%%%%%%%%%%%%%%%%%%%%%%%%%%%%%%%%%%    SECTION  %%%%%
\section{Conclusões}

\begin{itemize}

\item Conclusão 1

\item Conclusão 2


\end{itemize}




%%%%%%%%%%%%%%%%%%%%%%%%%%%%%%%%%%%
%%
%%                 AKNOWLEDGMENTS & BIBLIOGRAPHY
%%
%%%%%%%%%%%%%%%%%%%%%%%%%%%%%%%%%%%

{\footnotesize%


%%%%%%%%%%%%%%%%%%%%%%%%%%%%%%%%%%%%%%%%%%%    AGRADECIMENTOS  %%%%%
%\section*{Agradecimentos}
% Text







%%%%%%%%%%%%%%%%%%%%%%%%%%%%%%%%%%%%%%%%%%%    REFERÊNCIAS  %%%%%
\section*{Referências}
\vspace{5pt}

\begin{enumerate}[label={[\arabic*]}]

\item	Laugksch R., Scientific Literacy: A Conceptual Overview Science Education, 84, 71-94, 2000.

\item	Driver, R., Young people's images of science, Open University Press (Buckingham and Bristol, PA) 1996.

\item	Lopes J.B., Pinto J., Silva, A. Situação formativa: um instrumento de gestão do currículo capaz de
promover literacia científica. Enseñanza de las Ciencias, Número Extra VIII Congreso Internacional sobre
Investigación en Didáctica de las Ciencias, Barcelona, pp. 1624-1629 2009.



\end{enumerate}

}


\end{document}
