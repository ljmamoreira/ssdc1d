\documentclass[final,t]{beamer}
\usepackage[utf8]{inputenc}
\usepackage[portuges]{babel}
\usepackage[T1]{fontenc}
\usepackage{amsmath,pgfplots,times,filecontents}
\pgfplotsset{compat=1.8}

% poster template
\usepackage[orientation=portrait,size=a0,scale=1.4,debug]{beamerposter}
\usetheme{zurichposter}

% references
%\usepackage[bibstyle=authoryear, citestyle=authoryear-comp,%
%hyperref=auto]{biblatex}
\bibliography{references}

% document properties
\title{\LARGE Esquema de interpolação alternativo para problemas de difusão com
convecção}
\author{Luís J.M. Amoreira}
\institute{\ }

%block -> blue background title
%exampleblock -> green background title
%alertblock -> red background title
%------------------------------------------------------------------------------
\begin{document}

%twocolumns: width:0.488\linewidth, separated by \hfill
%singlecolumn: width: \linewidth
\begin{frame}{}
\begin{columns}[t]

%-----------------------------------------------------------------------------
%                                                                     COLUMN 1
% ----------------------------------------------------------------------------
\begin{column}{0.488\linewidth}
    %Introdução
    \begin{block}{Introdução --- Esquemas de interpolação}
        Nas aplicações do método dos volumes finitos, exprimem-se os valores do
        campo a determinar nas faces dos volumes de controle
        ($\phi^*_{i\pm1/2}$) como função dos seus valores nos pontos da malha de
        discretização ($\phi_i$) adotando um \emph{esquema de interpolação.}

        {\centering
            \begin{tikzpicture}[scale=4]
                \def\len{10}
                \def\hig{0.25}

                \draw [thick] (0.35*\len,0)--(0.83*\len,0);
                \node at (1.1*\len,0)
                    {$\phi^*_{i+1/2}=\phi^*_{i+1/2}(\phi_i,\phi_{i+1},\ldots)$};

                \draw (\len*0.4,\hig*0.8) --+(0,-1.6*\hig) node[below] {$i-1/2$} ;
                \fill (\len/2,0) circle(0.08) node [below] {$i$} node[above] {$\phi_i$};
                \draw (\len*0.6,\hig*0.8)node [above]{$\phi^*_{i+1/2}$} --+(0,-1.6*\hig) node[below] {$i+1/2$};
                \fill (\len*0.7,0) circle(0.08) node [below] {$i+1$} node[above]{$\phi_{i+1}$};
                \draw (\len*0.79,\hig*0.8) --+(0,-1.6*\hig);
            \end{tikzpicture}

        }
        Em problemas de difusão com convecção, a escolha de um esquema de
        interpolação deve ter em linha de conta a importância relativa dos
        transporte convectivo e difusivo, tradicionamente avaliada pelo 
        \emph{número de Peclet} $P$:
        \begin{equation*}
            \begin{aligned}
                |P|&\simeq0&\text{difusão dominante}\\
                |P|&>>1&\text{convecção dominante}
            \end{aligned}
        \end{equation*}
        O número de Peclet é proporcional à velocidade da convecção. Assim, tem
        o sinal da componente relevante da velocidade do escoamento.
    \end{block}

\end{column}


%-----------------------------------------------------------------------------
%                                                                     COLUMN 2
% ----------------------------------------------------------------------------
\hfill
\begin{column}{.488\linewidth}
    \begin{block}{Esquemas habitualmente usados}
        \begin{itemize}
            \item Esquema de derivação central (CDS): os valores dos campos nas
                faces dos VC são expressos como interpolações lineares dos
                valores nos pontos nodais\\
                Bons resultados sem escoamento, problemas de estabilidade com
                escoamentos pronunciados.
            \item Esquema \emph{upwind} (UWS): os valores dos campos nas  faces
                dos VC são tomados iguais aos valores nos pontos nodais a
                montante no escoamento\\
                Estável; soluções razoáveis só em situações de convecção
                claramente dominante
            \item Esquema híbrido (HYB): LDS se $|P|\simeq0$, UWS desprezando a
                difusão se $|P|>>1$.\\
                Estável; soluções razoáveis em diferentes regimes de escoamento;
                descontinuidade da solução para $P$ coincidente com o valor
                discriminante dos dois ramos.
                \rule[-13.5mm]{0mm}{1mm}
        \end{itemize}
    \end{block}
\end{column}
  
\end{columns}

\begin{columns}[t]
\begin{column}{\linewidth}
\begin{exampleblock}{Esquema alternativo --- Esquema de ajuste contínuo
    (CAS)}
    Esquema semelhante ao LDS, mas com coeficientes de interpolação ajustados
    em função da intensidade da convecção.
    \strut

    \begin{minipage}[c]{0.25\linewidth}
        \begin{align*}
            \phi^*_{i+1/2} &= f(P)\phi_i + [1-f(P)]\phi_{i+1}\\[5mm]
            f(P) &= \frac{1}{2}\left(1-\frac{P}{\xi+|P|}\right)
        \end{align*}
    \end{minipage}\hfill
    \begin{tikzpicture}[baseline={(current bounding box.center)}]
        \begin{axis}[samples=100,
                 domain=-3:3,
                 width=25.cm,
                 height=10cm,
                 xmin=-3, xmax=3,
                 ymin=-0.1,ymax=1.2,
			     no markers,
                 axis lines=left,
                 axis y line=middle,
                 axis x line = middle,
                 ytick = {0.5, 1},
                 xtick = {-2,-1,1,2},
                 xlabel = $P$,
                 ylabel = $f(P)$,
                 line width=2
                ]
                \addplot {0.5*(1+x/(0.3+abs(x)))};
        \end{axis}
    \end{tikzpicture}\hfill
    \begin{minipage}[c]{0.4\linewidth}
        Esta escolha de coeficientes de interpolação dá mais peso aos valores
        nos pontos nodais a montante, mas não anula completamente (como no UWS)
        a influência dos pontos a jusante.

        Sem escoamento ($v=0$, $P=0$), este esquema reduz-se ao LDS.
    \end{minipage}
\end{exampleblock}

\begin{block}{Estudo comparativo}
    \begin{minipage}[t]{0.6\linewidth}
        Comparam-se soluções obtidas com os diferentes esquemas para a difusão
        estacionária e unidimenional de calor, num fluido homogéneo com
        densidade $\rho$ e condutividade térmica $\Gamma$, que escoa com
        velocidade constante $v$, na ausência de fontes. Este problema tem
        solução analítica
        \begin{equation*}
            \phi(x) = \phi_0 + \frac{\phi_L-\phi_0}{e^{\mu}-1}
            \left(e^{\mu x}-1\right),\qquad \text{com }\mu=\rho v/\Gamma
        \end{equation*}
    \end{minipage}\hfill
    \begin{tikzpicture}[baseline={(current bounding box.north)}]
        \begin{axis}[samples=100,
                 domain=0:1,
                 width=20.cm,
                 height=10cm,
                 xmin=-0.02, xmax=1.1,
                 ymin=-0.1,ymax=1.3,
                 axis lines=left,
                 axis y line=middle,
                 axis x line = middle,
                 ytick = {0.001, 1},
                 yticklabels = {$\phi_L$, $\phi_0$},
                 xtick = {0,0.2,0.4,0.6,0.8,1},
                 xlabel = $x$,
                 ylabel = $\phi(x)$,
                 no markers,
                 line width=2
                ]
                \addplot [black] {1.0 - (exp(25*x)-1)/(exp(25.0)-1)};
        \end{axis}
        \node at (9,4) {%
            \begin{minipage}{2cm}
                \small
                \begin{align*}
                    v&=2,5\,\text{m/s}\\
                    \rho&=1,0\,\text{kg/m$^3$}\\
                    \Gamma&=0,1\,\text{kg\,m$^{-1}$\,s$^{-1}$}
                \end{align*}
            \end{minipage}
        };
    \end{tikzpicture}
    \hfill\strut
\end{block}

    % References
%    \begin{block}{Referências}
%        \vskip -0.8cm
%        \footnotesize
%        \begin{itemize}
%            \item Patankar
%            \item Versteeg \& Malalasekera
%            \item H.~Jing, C.~Li, B.~Zhou, European Conference on Fluid Dynamics
%                ECCOMAS CFD 2006
%        \end{itemize}
%        \normalsize
%        \vskip -0.8cm
%    \end{block}
\end{column}
\end{columns}
\begin{columns}[t]
    \begin{column}{0.316\linewidth}
        \begin{exampleblock}{Resultado: soluções}
                \centering
            \begin{tikzpicture}
                \small
                \begin{axis}[samples=100,
                         domain=0:1,
                         width=20.cm,
                         height=15cm,
                         xmin=-0.02, xmax=1.1,
                         ymin=-0.1,ymax=1.3,
                         axis y line=middle,
                         axis x line = middle,
                         ytick = {0.001, 1},
                         yticklabels = {$\phi_L$, $\phi_0$},
                         xtick = {0.1,0.3,0.5,0.7,0.9},
                         xlabel = $x$,
                         ylabel = $\phi(x)$,
                         line width=2,
                         axis lines=center,
                         legend style={at={(axis cs:0.03,0.03)},anchor=south west}
                        ]
                        \addplot [no markers] {1.0 - (exp(25*x)-1)/(exp(25.0)-1)};
                        \addplot [only marks, mark=x, mark size=8] table [x={x}, y={lds}]{phi.dat};
                        \addplot [only marks, mark=+, mark size=8] table [x={x}, y={uws}]{phi.dat};
                        \addplot [only marks, mark=o, mark size=8] table [x={x}, y={hyb}]{phi.dat};
                        \addplot [only marks, mark=square, mark size=8,blue] table [x={x}, y={cas}]{phi.dat};
                        \legend{{},LDS,UWS,HYB,CAS};
                \end{axis}
            \end{tikzpicture}
        \end{exampleblock}
    \end{column}\hfill
    \begin{column}{0.316\linewidth}
        \begin{exampleblock}{Resultados: erro como função de $P$}
            \begin{tikzpicture}
                \small
                \begin{axis}[%samples=100,
                         domain=0:1,
                         width=22.cm,
                         height=14.25cm,
                         xmin=0.0, xmax=45,
                         ymin=-0.001,ymax=0.2,
                         axis y line=middle,
                         axis x line = middle,
                         ytick = {0, 0.05,0.10,0.15},
                         yticklabels = {0,{0,05},{0,10},{0,15}},
                         xtick = {10,20,30,40},
                         xlabel = $P$,
                         ylabel = $\varepsilon$,
                         axis lines=center,
                         legend style={at={(axis cs:38,0.18)},anchor=north east}
                        ]
                        \addplot [only marks, mark=x, mark size=6] table [x={P}, y={lds}]{errtab.dat};
                        \addplot [only marks, mark=+, mark size=6] table [x={P}, y={uws}]{errtab.dat};
                        \addplot [only marks, mark=o, mark size=6] table [x={P}, y={hyb}]{errtab.dat};
                        \addplot [only marks, mark=square, mark size=8,blue] table [x={P}, y={cas}]{errtab.dat};
                        \legend{LDS,UWS,HYB,CAS};
                \end{axis}
            \end{tikzpicture}
        \end{exampleblock}
    \end{column}\hfill
    \begin{column}{0.316\linewidth}
        \begin{exampleblock}{Resultados: Taxa de convergência}
            \begin{tikzpicture}
                \small
                \begin{loglogaxis}[%samples=100,
                         %domain=0:1,
                         width=22.cm,
                         height=13.50cm,
                         xmin=0.001, xmax=1,
                         ymin=0.000005,ymax=0.5,
                         %axis y line=left,
                         %axis x line = middle,
                         %ytick = {0, 0.05,0.10,0.15},
                         %yticklabels = {0,{0,05},{0,10},{0,15}},
                         xtick = {0.001,0.01,0.1},
                         xlabel = $\delta x$,
                         ylabel = $\varepsilon$,
                         axis lines=center,
                         legend style={at={(axis cs:0.09,0.001)},anchor=north west}
                        ]
                        \addplot [only marks, mark=x, mark size=6] table [x={dx}, y={lds}]{conv_rate.dat};
                        \addplot [only marks, mark=+, mark size=6] table [x={dx}, y={uws}]{conv_rate.dat};
                        \addplot [only marks, mark=o, mark size=6] table [x={dx}, y={hyb}]{conv_rate.dat};
                        \addplot [only marks, mark=square, mark size=8,blue] table [x={dx}, y={cas}]{conv_rate.dat};
                        \legend{LDS,UWS,HYB,CAS};
                \end{loglogaxis}
            \end{tikzpicture}
        \end{exampleblock}
    \end{column}
\end{columns}
\begin{columns}[t]
    \begin{column}{0.488\linewidth}
        \begin{exampleblock}{Conclusões preliminares}
            \begin{enumerate}
                \item
                    O esquema proposto produz as soluções mais exactas de todos
                    os esquemas testados, qualquer que seja o regime de
                    escoamento ou o passo de integração
                \item
                    O esquema proposto apesenta uma taxa de convergência muito
                    elevada, semelhante ao LDS (de segunda ordem em malhas
                    homogéneas)
            \end{enumerate}
        \end{exampleblock}
    \end{column}
    \begin{column}{0.488\linewidth}
        \begin{exampleblock}{Futuros desenvolvimentos}
            \begin{itemize}
                \item Como se comporta o esquema proposto em problemas que
                    incluam a presença de fontes?
                \item Como se comporta o esquema proposto em malhas não
                    homogéneas?
                \item Como se comporta o esquema proposto em problemas 2- e
                    3-dimen\-sionais?
            \end{itemize}
            \strut
        \end{exampleblock}
    \end{column}
\end{columns}
\begin{columns}[t]
    \begin{column}{\linewidth}
        \small
        \begin{block}{Referências}
            \begin{itemize}
                \item Patankar
                \item Verteeg \& Malalasekera
                \item H.~Jing, C.~Li, B.~Zhou, European Conference on Fluid Dynamics
                    ECCOMAS CFD 2006
            \end{itemize}
        \end{block}
    \end{column}
\end{columns}
\end{frame}

\end{document}
