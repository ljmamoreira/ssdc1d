\documentclass[final,t]{beamer}
\usepackage[utf8]{inputenc}
\usepackage[portuges]{babel}
\usepackage[T1]{fontenc}
\usepackage{amsmath}
\usepackage{times}

% poster template
\usepackage[orientation=portrait,size=a0,scale=1.4,debug]{beamerposter}
\usetheme{zurichposter}

% references
%\usepackage[bibstyle=authoryear, citestyle=authoryear-comp,%
%hyperref=auto]{biblatex}
\bibliography{references}

% document properties
\title{\LARGE Esquema de interpolação alternativo para problemas de difusão com
convecção}
\author{Luís J.M. Amoreira}
\institute{\ }

%block -> blue background title
%exampleblock -> green background title
%alertblock -> red background title
%------------------------------------------------------------------------------
\begin{document}

%twocolumns: width:0.488\linewidth, separated by \hfill
%singlecolumn: width: \linewidth
\begin{frame}{}
    \begin{block}{Introdução}
    \end{block}
\begin{columns}[t]

%-----------------------------------------------------------------------------
%                                                                     COLUMN 1
% ----------------------------------------------------------------------------
\begin{column}{0.488\linewidth}
    %Introdução
    \begin{block}{Difusão com convecção}
        A equação de difusão de uma quantidade conservada com densidade mássica
        $\phi$, num meio material com densidade $\rho$, em movimento (difusão com
        convecção) com velocidade $v$, em situações estacionárias e
        unidimensionais é [refs]
        \begin{equation*}
            \frac{d}{dx}(\rho v \phi)=
            \frac{d}{dx}\left(\Gamma\frac{d\phi}{dx}\right) + S(x),
        \end{equation*}
        onde $\Gamma$ é o coeficiente de difusão e $S$ é a densidade de fonte da
        dita quantidade.

        \vspace{5mm}
        Numa abordagem numérica com o método dos volumes finitos, integra-se
        esta equação em subintervalos (\emph{volumes de controle,} VC) que
        constituem uma partição do domínio de integração:

        {\centering
            \begin{tikzpicture}[scale=2]
            \def\len{10}
            \def\hig{0.5}
            \fill [gray!15] (\len*0.4,-\hig*0.8) rectangle (\len*0.6,\hig*0.8);
            \draw [thick] (0,-\hig) --+ (0,2*\hig);
            \draw [thick] (0,0) node[left] {$\phi_A$} -- (0.12*\len,0);
            \draw (0.12*\len,0)--++(240:0.2)--+(60:0.4);
            \draw (0.15*\len,0)--++(240:0.2)--+(60:0.4);

            \draw [thick] (0.15*\len,0)--(0.87*\len,0);
            \draw [thick] (0.9*\len,0)--(\len,0) node[right]{$\phi_B$};
            \draw (0.87*\len,0)--++(240:0.2)--+(60:0.4);
            \draw (0.9*\len,0)--++(240:0.2)--+(60:0.4);

            \draw (\len*0.21,\hig*0.8) --+(0,-1.6*\hig);
            \fill (\len*0.3,0) circle(0.08) node [below] {$W$};
            \draw (\len*0.4,\hig*0.8) --+(0,-1.6*\hig) node[below] {$w$};
            \fill (\len/2,0) circle(0.08) node [below] {$P$};
            \draw (\len*0.6,\hig*0.8) --+(0,-1.6*\hig) node[below] {$e$};
            \fill (\len*0.7,0) circle(0.08) node [below] {$E$};
            \draw (\len*0.79,\hig*0.8) --+(0,-1.6*\hig);

            \draw [thick] (\len,-\hig) --+ (0,2*\hig);
        \end{tikzpicture}

        }
        Para o VC destacado na figura (com centro no ponto $P$),
        este procedimento conduz a
        \begin{equation*}
            (\rho v \phi)_e - (\rho v \phi)_w \simeq
            \Gamma_e\left(\frac{d\phi}{dx}\right)_e-
            \Gamma_w\left(\frac{d\phi}{dx}\right)_w + S_P\delta x_P
        \end{equation*}
        Juntando a esta equação as que de modo semelhante obtêm para os restante
        VC, resulta um sistema de equações algébricas lineares para os valores
        de $\phi$ nos pontos nodais $W$, $P$, $E$, etc, que pode ser resolvido
        numericamente.

        \vspace{5mm}
        Um ingrediente indispensável neste método é a definição de uma regra
        (\emph{esquema de interpolação}) para estimar os valores do campo $\phi$
        e do seu gradiente $d\phi/dx$ nas faces ($e$, $w$, etc.) dos VC a partir
        dos valores de $\phi$ nos pontos nodais $W$, $P$, $E$, etc.

        \vspace{5mm}
        A qualidade de uma dado esquema de interpolação depende, em geral, da
        importância relativa dos dois fenómenos de transporte envolvidos, a
        difusão e a convecção. Tradicionalmente, essa importância é avaliada
        pelo chamado \emph{número de Peclet},
        \begin{equation*}
            P=\frac{\rho v \delta x}{\Gamma}
        \end{equation*}
    \end{block}

\end{column}


%-----------------------------------------------------------------------------
%                                                                     COLUMN 2
% ----------------------------------------------------------------------------
\hfill
\begin{column}{.488\linewidth}
  
    % Conclusions
    \begin{block}{Esquemas tradicionais e esquema proposto}
        \begin{itemize}
            \item Joint formulation of LP and QP relaxation $\rightarrow$ LPQP.
            \item LPQP solved by a message-passing
            algorithm for modified unary potentials.
            \item Get a smooth objective for free. Key to fast convergence.
            \item Competitive results in terms of MAP state found.
        \end{itemize}
    \end{block}

    % References
    \begin{block}{Referências}
        \vskip -0.8cm
        \footnotesize
        \begin{itemize}
            \item Patankar
            \item Versteeg \& Malalasekera
            \item H.~Jing, C.~Li, B.~Zhou, European Conference on Fluid Dynamics
                ECCOMAS CFD 2006
        \end{itemize}
        \normalsize
        \vskip -0.8cm
    \end{block}


\end{column}


\end{columns}

\begin{columns}[t]
\begin{column}{\linewidth}
\begin{exampleblock}{Análise comparativa}
\begin{itemize}
\item LDS
\item UWS
\item HYB
\item CAS
\end{itemize}
\end{exampleblock}

\end{column}
\end{columns}
\end{frame}

\end{document}
