\documentclass[final,t]{beamer}
\usepackage[utf8]{inputenc}
\usepackage[portuges]{babel}

% poster template
\usepackage[orientation=portrait,size=a0,scale=1.4,debug]{beamerposter}
\usetheme{zurichposter}

% references
%\usepackage[bibstyle=authoryear, citestyle=authoryear-comp,%
%hyperref=auto]{biblatex}
\bibliography{references}

% document properties
\title{\LARGE Esquema de interpolação alternativo para problemas de difusão com
convecção}
\author{Luís J.M. Amoreira}
\institute{\ }

%block -> blue background title
%exampleblock -> green background title
%alertblock -> red background title
%------------------------------------------------------------------------------
\begin{document}

\begin{frame}{}
\begin{columns}[t]


%-----------------------------------------------------------------------------
%                                                                     COLUMN 1
% ----------------------------------------------------------------------------
\begin{column}{.488\linewidth}

    % MAP problem
    \begin{exampleblock}{Energy Minimization and MAP Inference}
        See http://pletscher.org/papers/pletscher2012lpqp-poster.pdf for a
        full version of this poster.
    \end{exampleblock}

\end{column}


%-----------------------------------------------------------------------------
%                                                                     COLUMN 2
% ----------------------------------------------------------------------------
\hfill
\begin{column}{.488\linewidth}
  
    % Conclusions
    \begin{alertblock}{Conclusions}
        \begin{itemize}
            \item Joint formulation of LP and QP relaxation $\rightarrow$ LPQP.
            \item LPQP solved by a message-passing
            algorithm for modified unary potentials.
            \item Get a smooth objective for free. Key to fast convergence.
            \item Competitive results in terms of MAP state found.
        \end{itemize}
    \end{alertblock}

    % References
    \begin{block}{References}
        \vskip -0.8cm
        \footnotesize
        \begin{itemize}
            \item Patankar
        \end{itemize}
        \normalsize
        \vskip -0.8cm
    \end{block}


\end{column}


\end{columns}

\begin{columns}[t]
\begin{column}{\linewidth}
\begin{exampleblock}{Análise comparativa}
\begin{itemize}
\item LDS
\item UWS
\item HYB
\item CAS
\end{itemize}
\end{exampleblock}

\end{column}
\end{columns}
\end{frame}

\end{document}
